% W'keit/Statistik Zusammenfassung aus dem Informatikstudium an der ETH Zuerich
% basierend auf der Vorlesung von Prof. Dr. Thorsten Rheinlaender und Prof. Dr.
% Philipp Schoenbucher
% Copyright (C) 2003  Patrick Pletscher

%This program is free software; you can redistribute it and/or
%modify it under the terms of the GNU General Public License
%as published by the Free Software Foundation; either version 2
%of the License, or (at your option) any later version.

%This program is distributed in the hope that it will be useful,
%but WITHOUT ANY WARRANTY; without even the implied warranty of
%MERCHANTABILITY or FITNESS FOR A PARTICULAR PURPOSE.  See the
%GNU General Public License for more details.

%You should have received a copy of the GNU General Public License
%along with this program; if not, write to the Free Software
%Foundation, Inc., 59 Temple Place - Suite 330, Boston, MA  02111-1307, USA.
\documentclass[10pt, a4paper, twocolumn]{scrartcl}

\usepackage{german}
\usepackage{amsmath}
\usepackage{amsfonts}
\usepackage[pageanchor=false,colorlinks=true,urlcolor=black,hyperindex=false]{hyperref}

\textwidth = 16.5 cm
\textheight = 25 cm
\oddsidemargin = 0.0 cm
\evensidemargin = 0.0 cm
\topmargin = 0.0 cm
\headheight = 0.0 cm
\headsep = 0.0 cm
\parskip = 0 cm
\parindent = 0.0cm

% Tiefe des Inhaltsverzeichnisses
\setcounter{secnumdepth}{2}
\setcounter{tocdepth}{2}

\title{Wahrscheinlichkeitsrechnung und Statistik - Zusamenfassung}
\author{Patrick Pletscher}
\begin{document}

\maketitle

\section{Wahrscheinlichkeiten}

\subsection{Ereignisraum}

Der \textit{Ereignisraum $\Omega$} umfasst alle m"oglichen Ausg"ange eines Zufallsexpermiments.\\
Ein \textit{Elementarereignis} $\Omega$ ist ein Element $\omega\:\in\:\Omega$\\
Ein Ereignis A ist eine Teilmenge von $\Omega$, d.h. eine Kombination von Elementarereignissen $A \subset \Omega$\\

$A^c$ (Komplement) ist das Ereignis, dass A nicht eintritt.\\

$\mathcal{A}$ ist die Klasse der beobachteten Ereignisse. Falls $\Omega$ endlich ist, dann ist $\mathcal{A}$ die Menge aller Teilmengen  von $\Omega$, d.h. die Potenzmenge.

\subsection{Das Wahrscheinlichkeitsmass}

$P:\mathcal{A}\rightarrow[0,1]$\\
P(A)$\doteq$ 'die Wahrscheinlichkeit, dass A eintritt'

\subsubsection{Axiome der Wahrscheinlichkeitstheorie}
\begin{description}
 \item [A1] $0\leq P(A) \leq 1$ f"ur alle $A \subset \Omega$
 \item [A2] $P(\Omega )=1$
 \item [A3] Sei $A_1,A_2,\ldots$ eine Folge disjunkter Ereignisse, dann $P(\cup^\infty_{i=1})=\sum^\infty_{i=1}P(A_i)$
\end{description}

\subsubsection{Weitere Rechenregeln}

\begin{description}
 \item $P(A^c)=1-P(A)$
 \item $P(\emptyset)=0$
 \item $A \subset B \Rightarrow P(A)\leq P(B)$
 \item $P(A \cup B)=P(A)+P(B)-P(A \cap B)$
 \item $P(A\cup B)=1-(\bar{P(A)}+\bar{P(B)})$
 \item und $\rightarrow$ Multiplikation
\end{description}


\subsection{Berechnung von W'keiten in endlichen R"aumen}

$P(A)=\frac{|A|}{|\Omega|}$ 

\subsubsection{Exkurs in die Kombinatorik}

\textbf{Permutationen ohne Zur"ucklegen}

Aus n Objekten sind $k \leq n$ herauszugreifen, wobei die Reihenfolge eine Rolle spielen soll.\\

$\sharp$ M"ogl. $=\frac{n!}{(n-k)!}$\\\\

\textbf{Permutationen mit Zur"ucklegen}

Gegeben sind n Objekte. Wieviele Folgen der L"ange k k"onnen gebildet werden, falls jedes Objekt beliebig oft gew"ahlt werden darf.\\

$\sharp$ M"ogl. $=n^k$\\\\

\textbf{Kombinationen ohne Zur"ucklegen}

Gegeben eine Menge mit n Elementen. Wieviele Teilmengen mit $k\leq n$ Elementen kann man daraus bilden?\\

$\sharp$ M"ogl. $=\frac{n!}{(n-k)!k!}=\binom{n}{k}$\\\\

Bsp: Lotto (6 aus 45)\\
$A=\{6 Richtige\}=\binom{45}{6}$\\
$B=\{4 Richtige\}=\binom{6}{4}\binom{39}{2}$

\subsection{Bedingte W'keiten}

Seien A,B Ereignisse, $P(A)>0$\\
\textbf{Def.:} Die \textit{bedingte W'keit}, dass A gegeben B eintritt, ist\\
$P(B|A)=\frac{P(A\cap B)}{P(A)}$\\\\

\textbf{Multiplikationssatz}\\
Sei $P(A) > 0$: Dann\\
$P(A\cap B)=P(B|A)\cdotp P(A)$\\\\

\textbf{Satz von der totalen W'keit}\\
Gegeben seien Ereignisse $A_1,\ldots,A_n$ mit $P(A_i)>0$.\\
\textbf{Def.:} Die $A_i$ bilden eine Zerlegung von $\Omega$, falls
\begin{enumerate}
 \item $A_i \cap A_j = \emptyset$ f"ur $i \neq j$, paarw. disj.
 \item $\Omega = \bigcup\limits^n_{i=1}A_i$
\end{enumerate}
Dann gilt f"ur $B \subset \Omega$ dass\\
$P(B)=\sum\limits^n_{i=1}P(B|A_i)\cdotp P(A_i)$\\\\

\textbf{Satz von Bayes}\\
$P(A|B)=\frac{P(A\cap B)}{P(B)}=\frac{P(B|A)\cdotp P(A)}{P(B|A)\cdotp P(A)+P(B|A^C)\cdotp P(A^C)}$\\\\

\textbf{Allgemeine Formel von Bayes}\\
$P(A_k|B)=\frac{P(B|A_k) P(A_k)}{\sum\limits^n_{i=1} P(B|A_i)P(A_i)}$

\subsection{Unabh"angigkeit}

Seien $A,B \subset \Omega$ zwei Ereignisse.\\
\textbf{Def.} A und B sind \textit{unabh"angig}, falls $P(A\cap B)=P(A)P(B)$\\
Falls $P(A)>0$:\\
A,B unabh. $\Leftrightarrow P(B|A)=P(B)$\\
Falls $P(B)>0$:\\
A,B unabh. $\Leftrightarrow P(A|B)=P(A)$

\subsubsection{Allgemeine Definition der Unabh"angigkeit}

n Ereignisse $A_1,\ldots,A_n$ heissen \textit{unabh"angig}, falls f"ur \textit{jede} Wahl von je m Ereignissen\\
$A_{k1},\ldots,A_{km},\{k_1,\ldots,k_m\}\subset\{1,\ldots,n\}$,\\
stets gilt:\\
$P(A_{k1}\cap\ldots\cap A_{km})=P(A_{k1})\cdotp \ldots \cdotp P(A_{km})$

\section{Zufallsvariable}

\subsection{Begriff der ZV}

Es sei $\Omega$ ein Ereignisraum. Eine ZV auf $\Omega$ ist eine Funktion $\Omega \rightarrow \mathbb{R}$.\\
Wir nennen eine ZV diskret, falls sie endliche oder abz"ahlbar viele Werte annimmt.

\subsection{Wahrscheinlichkeit- und Verteilungsfunktion}

{\bf Def.} Die W'keitsfunktion der diskreten ZV X ist die Funktion\\
$p(x)=\left\{
 \begin{array}{ll}
  P(X=x)    	& $falls $ x \in \omega \\
  0   		& $sonst$ 
 \end{array}
\right.$

$\omega$: Wertebereich von X\\\\

Verteilungsfunktion F: $F(x)=P(X\leq x)$\\
F"ur diskrete ZV ist\\
$F(x)=\sum p(x_i)$\\
$x_i:x_i\in\omega,x_i\leq x$\\\\

Die Verteilungsfunktion ist:
\begin{itemize}
 \item rechtsstetig
 \item $\lim_{x\rightarrow -\infty}F(X)=0,\:\lim_{x\rightarrow +\infty}F(X)=1$
\end{itemize}


\subsection{Einige wichtige diskrete Verteilungen}

\begin{enumerate}
 \item Uniforme Verteilung (Gleichverteilung)\\
  $\omega=\{x_1,\ldots,x_n\}$\\
  $p(x_i)=P(X=x_i)=\frac{1}{n}$
  Bsp: W"urfeln mit einem W"urfel
 \item Bernoulli-Verteilung\\
  X nimmt nur die Werte 0 und 1 an\\
  $P(X=1)=p$\\
  $P(X=0)=1-p$\\
  $P(X)=p^x\cdotp(1-p)^{1-x}\:\:,x=0,1$\\
  $X\sim Be(p)$
 \item Binomialverteilung\\
  X=''Anzahl Erfolge bei n Versuchen''\\
  $p(k)=\binom{n}{k}p^k(1-p)^{n-k}$\\
  $X\sim B(n,p)$\\
  Approx. durch Poisson\\
  Multinomialverteilung\\
  \scriptsize $P[X_1=x_1,X_2=x_2,X_3=x_3]=\frac{n!}{x_1!x_2!x_3!}p_1^{x_1}p_2^{x_2}p_3^{x_3}$\normalsize
 \item Die geometrische Verteilung\\
  X=''Anzahl der Versuche, bis ein Erfolg eintritt''\\
  $p(k)=P(X=k)=(1-p)^{k-1}p$
 \item Die negativbinomiale Verteilung\\
  X=''Anzahl Versuche, bis ich r-mal erfolgreich bin''\\
  $P(X=k)=\binom{k-1}{r-1}p^r(1-p)^{k-r}$
 \item Die hypergeometrische Verteilung\\
  n Gegenst"ande in einer Urne, r vom Typ I und n-r vom Typ II, ich ziehe m davon (ohne Zur"ucklegen)
  $P(X=k)=\frac{\binom{r}{k}\binom{n-r}{m-k}}{\binom{n}{m}}$\\
  Bsp: Lotto, 6 aus 45, W'keit eines Vierers:\\
  n=45, r=6, m=6, k=4
 \item Die Poisson-Verteilung\\
  X=''Anzahl Ereignisse in einem Zeitintervall(Anrufe, gedruckte Files)''\\
  X ist Poisson-verteilt mit Parameter $\lambda$, falls\\
  $P(X=k)=\frac{\lambda^k e^{-\lambda}}{k!}\:\:k=0,1,\ldots$\\
  Wertebereich $W={1,\ldots,\infty}$\\
  $\lambda=E[X]=n\cdotp p$
\end{enumerate}

\subsection{Stetige Zufallsvariablen}

Sind ZV, die Werte in einem Intervall W annehmen k"onnen.\\
z.B.: $W=\mathbb{R},\:W=\mathbb{R}^+,\:W=[0,1]$\\
\textbf{Definition}\\
Sei X eine ZV mit \textit{Verteilungsfunktion oder W'keitsfunktion} $F(x)=P[X\leq x]$.\\
Falls es eine Funktion f(x) gibt, so dass $F(x)=\int^x_{-\infty}f(y)dy$ f"ur alle $x \in W$, dann heisst $f(x)$ die \textit{Dichte} von X.\\\\

\textbf{Eigenschaften}
\begin{itemize}
 \item $f(x)\geq 0$ f"ur alle x
 \item $f(x)$ ist stetig oder st"uckweise stetig
 \item $\int\limits^{\infty}_{-\infty}f(x)dx=1$, weil $\lim\limits_{x\rightarrow\infty}F(x)=1$
 \item $a<b:\:P[a<X\leq b]=P[X\leq b]-P[X\leq a]=F(b)-F(a)=\int\limits^b_a f(x)dx$
 \item $P[X=a]=0$
\end{itemize}

\textbf{Satz}\\
An allen Stellen, an denen $f(x)$ stetig ist, gilt:\\
$F'(x)=f(x)$\\\\

\textbf{Beispiele}
\begin{enumerate}
 \item Gleichverteilung (Uniform Distribution): $X\sim U(0,1)$\\
  $f(x)=\left\{
  \begin{array}{l}
   1$ f"ur $ x \in [0,1] \\
   0$ sonst$ 
  \end{array}
  \right.$\\
  $F(x)=\left\{
  \begin{array}{l}
   0$ f"ur $ x < 0 \\
   x$ f"ur $x\in[0,1]\\
   1$ f"ur $x>1
  \end{array}
  \right.$
 \item Exponentialverteilung $X\sim Exp(\lambda)$\\
  $\lambda > 0$\\
  $f(x)=\left\{
  \begin{array}{l}
   \lambda e^{-\lambda x}$ f"ur $ x>0\\
   0$ sonst$ 
  \end{array}
  \right.$\\
  $F(x)=\left\{
  \begin{array}{l}
   \int\limits^x_0 f(y)dy=\int\limits^x_0\lambda e^{-\lambda x}dy=1-e^{-\lambda x}$ f"ur $ x\geq0\\
   0$ f"ur $x<0 
  \end{array}
  \right.$\\
  Die Exp-Verteilung ist ged"achnislos.\\
  Ein Ankunftsprozess, bei dem die Zeiten zwischen den Ank"unften Exponentiell-verteilt ist, heisst Poisson-Prozess. $\lambda$ heisst dann die (Ankunfts)Rate des Poisson-Prozess.
  \item Die Normalverteilung / Gauss-Verteilung $X\sim N(\mu,\sigma^2)$\\
   \begin{displaymath}
    f(x)=\frac{1}{\sqrt{2\Pi}\sigma}exp\{-\frac{1}{2}(\frac{x-\mu}{\sigma})^2\}
   \end{displaymath}
   $\mu=$ ''Mean'', Mittelwert, $-\infty < \mu < \infty$\\
   $\sigma=$ Standardabweichung, $\sigma > 0$\\
   $\sigma^2=$ Varianz\\
   $F(x)=\int\limits^x_{-\infty}f(y)dy$: keine geschlossene Form, aber Tabellen\\
   Standardisierte NV:\\
   $\mu=0,\:\sigma=1$\\
   Wenn $X\sim N(\mu,\sigma^2)$ dann ist $\frac{x-\mu}{\sigma}\sim N(0,1)$\\
   $P[X\leq x']=P[\frac{x-\mu}{\sigma}\leq\frac{x'-\mu}{\sigma}]=\Phi(\frac{x'-\mu}{\sigma})$\\
   $\Phi(-x)=1-\Phi$
\end{enumerate}

\subsection{Transformation von ZV}

Sei X eine ZV mit Verteilungsfkt. $F_X(x)$ und Dichte $f_X(x)$.\\
Gesucht: Verteilung und Dichte von $Y=g(X)$\\\\

\textbf{Satz}\\
Wenn $X \sim N(\mu,\sigma^2)$ und $Y=aX+b$, dann gilt $Y\sim N(a\mu+b,a^2\sigma^2)$\\\\

{\bf Satz}\\
Sei X stetig, mit Dichte $f_X(x)$, Verteilung $F_X(x)$. Sei $Y=g(X)$, mit g diff'bar, streng monoton steigend/fallend auf einem Intervall I, wobei $f_X(x)=0$ f"ur $x\notin I$ ($I=\mathbb{R}$ ist zugelassen).\\
Dann ist die Dichte von Y\\
$f_Y(y)=\left\{
\begin{array}{l}
 f_X(g^{-1}(y))\cdotp|\frac{d}{dy}g^{-1}(y)|\\
 0$ sonst$ 
\end{array}
\right.$\\
wobei $g^{-1}$ die Umkehrfkt. von g und $y\in\{g(x)|x\in I\}$\\\\

und die Verteilungsfkt. von Y ist\\
$F_Y(y)=\left\{
\begin{array}{l}
 F_X(g^{-1}(y))$ f"ur g steigend$\\
 1-F_X(g^{-1}(y))$ f"ur g fallend$
\end{array}
\right.$\\

$y\in\{g(x)|x\in I\}$\\\\

\textbf{Lognormale Verteilung}\\
$X\sim N(\mu,\sigma^2),\:Y=e^x$\\
$g(x)=e^x,\:g^{-1}(y)=\ln y,\:I=\mathbb{R}$\\\\

\textbf{Satz}\\
Sei $U\sim U(0,1)$.\\
Sei $F(x)$ eine stetige, streng monoton wachsende Verteilungsfunktion. Setze\\
$X:=F^{-1}(U)$\\
Dann gilt:\\
$P[X\leq x]=F(x)$\\\\

\textbf{Definition}
\begin{itemize}
 \item Der Wert $F^{-1}(p)$ (f"ur $p\in(0,1)$) heisst das p-Quantil der Verteilung F
 \item $F^{-1}(0.5)$ heisst der Median der Verteilung F.
\end{itemize}

\section{Gemeinsame Verteilung mehrerer ZV}

Seien $X_1,\ldots,X_n$ Zufallsvariablen.\\

Dann ist $F(x_1,\ldots,x_n)=P[X_1\leq x_1,\ldots,X_n\leq x_n]$

\subsection{Stetige ZV}

Falls wir $F(x_1,\ldots,x_n)$ folgendermassen darstellen k"onnen $F(x_1,\ldots,x_n)=\int\limits^{x_1}_{-\infty}\ldots\int\limits^{x_n}_{-\infty}f(y_1,\ldots,y_n)dy_n\ldots dy_1$ dann heisst $f(x_1\ldots x_n)$ die Dichte von $(X_1\ldots X_n)$\\\\

\textbf{Eigenschaften}
\begin{itemize}
 \item $P[(X_1\ldots X_n)\in A]=\iint\limits_A f(\vec{x})d\vec{x}=\int\ldots\int\limits_{(x_1\ldots x_n)\in A}f(x_1,\ldots,x_n)dx_1\ldots dx_1$
 \item $\int\limits^\infty_{-\infty}\ldots\int\limits^\infty_{-\infty}f(x_1,\ldots,x_n)dx_n\ldots dx_1=1$\\
 \item $f(x_1,\ldots,x_n=\frac{\partial^n}{\partial x_1\ldots \partial x_n}F(x_1 \ldots x_n)$
\end{itemize}

\subsection{Randverteilungen}

Gegeben sei die gemeinsame Verteilung von X und Y:$F(x,y)$. Die Randverteilung von X ist:\\
$F_X(x)=P[X\leq x]=P[X\leq x;\:Y\in(-\infty,\infty)]=\lim\limits_{y\rightarrow \infty}F(x,y)$\\\\

Diskrete ZV: $Y\in\{y_1,y_2,\ldots\}$\\
Die W'keitsfkt. der Randverteilung von X ist\\
$p_X(x)=\sum\limits_j p(x,y_j)$\\\\

Stetige ZV: Die Dichte der RV von X ist\\
$f_X(x)= \frac{d}{dx}F_X(x)=\frac{d}{dx}\int\limits^x_{-\infty}[\int\limits^\infty_{-\infty}f(x',y')dy']dx'=\int\limits^\infty_{-\infty}f(x,y')dy'$\\\\

\textit{Beispiel}\\
$f(x,y)=\left\{
\begin{array}{l}
 6(x-y)$ f"ur $ 0\leq y \leq x \leq 1\\
 0$ sonst$ 
\end{array}
\right.$\\

$f_X(x)=\int\limits^x_0 f(x,y)dy$\\
$f_Y(y)=\int\limits^1_y f(x,y)dx$


\subsection{Unabh"angigkeit von ZV}

Die ZV $X_1,\ldots,X_n$ sind unabh"angig, falls $F(x_1,\ldots,x_n)=F_{X_1}(x_1)\cdotp F_{X_2}(x_2)\cdotp\ldots\cdotp F_{X_n}(x_n)$ f"ur alle $(x_1 \ldots x_n)\in\mathbb{R}^n$

\begin{enumerate}
 \item Diskreter Fall: $(X_1\ldots X_n)$ unabh. $\Leftrightarrow p(x_1\ldots x_n)=P_{X_1}\cdotp\ldots\cdotp P_{X_n}(x_n)$
 \item Stetiger Fall: $f(x_1\ldots x_n)=f_{X_1}(x_1)\cdotp\ldots\cdotp f_{X_n}(x_n)$
\end{enumerate}

X,Y unabh"angig $(F(x,y)=F_X(x)\cdotp F_Y(y))\Leftrightarrow f(x,y)=f_X(x)\cdotp f_Y(y)$


\subsubsection{Wichtige Mehrdimensionale Verteilungen}

\begin{enumerate}
 \item die mehrdimensionale Normalverteilung (stetige Verteilung)\\
  Dichte:\\
  $f(x,y)=\frac{1}{2\Pi\sigma_X\sigma_Y\sqrt{1-\rho^2}}\cdotp \\ exp\{-\frac{1}{2(1-\rho^2)[(\frac{x-\mu_X}{\sigma_X})^2+(\frac{y-\mu_Y}{\sigma_Y})^2-\frac{2\rho(x-\mu_X)(y-\mu_Y)}{\sigma_X\sigma_Y}]}\}$\\
  $-1<\rho<+1$ Korrelation zw. X und Y.\\
  Randverteilung\\
  $f_X(x)=\frac{1}{\sigma_X\sqrt{2\Pi}}exp(-\frac{1}{2}(\frac{x-\mu_X}{\sigma_X})^2)$\\
  Unabh"angigkeit von X und Y, genau dann, wenn $\rho=0$
 \item Die Multinomiale Verteilung (diskret)\\
  Es werden n unabh"angige Experimente durchgef"uhrt. Es gibt jeweils r m"ogliche Ergebnisse mit W'keit $p_1,\ldots,p_r$. $\sum^n_{i=1}p_i=1$.\\
  Sei $N_i$ die Anzahl der Ergebnisse ''i''.\\
  Die Vtlg. von $[N_1,\ldots,N_r]$ heisst Multinomialvtlg.\\
  $p(N_1=n_1,\ldots,N_r=n_r)=\binom{n}{n_1\cdots n_r}p_1^{n_1}\cdots p_r^{n_r}$\\
  $\binom{n}{n_1\cdots n_r}=\frac{n!}{n_1!\cdots n_r!}$: Multinomialkoeffizient\\
  RV von $N_i$:\\
  $p_{N_i}(n_i)=P[N_i=n_i]=\binom{n}{n_i}p_i^{n_i}(1-p_i)^{n-n_i}$\\
  $p(n_1\ldots n_r)\neq\prod\limits^r_{i=1}p_{N_i}(n_i)\Leftrightarrow$ keine Unabh.
 \item Mehrdim. Verteilungen mit U(0,1)-RV\\
  $F(x,y)=exp(-[(-\ln x)^\beta+(-\ln y)^\beta]^{\frac{1}{\beta}}$\\
  $x,y\in]0,1],\beta\geq 1$\\
  RV: $F_X(x)=F(x,y=1)=x$: U(0,1)-Vrtlg. (Y genauso)
\end{enumerate}

\subsection{Bedingte Verteilungen}

\subsubsection{Diskrete ZV}

X,Y ZV diskret mit gemeinsamer W'keitsfkt. p(x,y)\\
Def. Die bedingte W'keitsfkt. von X unter der Bedingung, dass Y=y ist, ist $p_{X|Y}(x|y)=P[X=x|Y=y]=P[X=x,Y=y]/P[Y=y]$\\\\

Bem:
\begin{itemize}
 \item $\sum\limits_i p_{X|Y}(x_i,y)=\frac{\sum\limits_i p(x_i,y)}{p_Y(y)}=1$
 \item Wenn $p_{X|Y}(x|y)=p_X(x)$ f"ur alle x,y, dann sind X und Y unabh.
\end{itemize}

\subsubsection{Stetige ZV}

Seien X,Y stetige ZV mit gem. Dichte $f(x,y)$\\
Def. Die bedingte Dichte von Y, gegeben X=x, ist\\
$f_{Y|X}(y|x)=\frac{f(x,y)}{f_X(x)}$, wenn $0<f_X(x)<\infty$ sont $f_{Y|X}(y|x)=0$\\\\

Bem:
\begin{itemize}
 \item $\int\limits^\infty_{-\infty}f_{Y|X}(y|x)dy=1$
 \item $f_{Y|X}(y|x)=f_Y(y)$ f"ur alle x,y $\Leftrightarrow$ X und Y sind unabh.
\end{itemize}


\subsection{Funktionen von ZV}

X,Y ZV mit bekannter gem. Vtlg.\\

Was ist die Vtlg. der Summe X+Y?\\
\begin{enumerate}
 \item Diskreter Fall\\
  $Z=X+Y$\\
  $p_Z(z)=\sum\limits_i p(x_i,z-x_i)$,\\
  Falls X,Y unabh"angig:\\
  $p_Z(z)=\sum\limits_i p_X(x_i)p_Y(z-x_i)$ (Faltung von X,Y)
 \item stetiger Fall\\
  X,Y Dichtefkt. f(x,y) sind gegeben. Z=X+Y\\
  $f_Z(z)=\int\limits^{+\infty}_{-\infty} f(x,z-x)dx$\\
  Falls X,Y unabh"angig:\\
  $f_Z(z)=\int\limits^{+\infty}_{-\infty}f_X(x)f_Y(z-x)dx$
\end{enumerate}


\section{Erwartungswert}

\subsection{Definition und Eigenschaften}

\textbf{Diskrete ZV}\\
X sei diskrete ZV mit W'keitsfunktion p(x). Dann ist der EW von X def. durch $E[X]=\sum\limits_i x_ip(x_i)$ falls die Reihe absolut konvergiert.

\begin{enumerate}
 \item Bernoulli Verteilung ($X\sim Be(p)$)\\
  $E[X]=1\cdotp p + 0\cdotp(1-p)=p$
 \item Geometrische Verteilung ($X\sim Nb(1,p)$)\\
  $E[X]=\frac{1}{p}$
 \item Poisson Verteilung ($X\sim P_{\lambda}(x)$)\\
  $E[X]=\lambda\sum\limits^\infty_{k=1}\frac{e^{-\lambda}\lambda^k}{k!}$
\end{enumerate}

\textbf{Stetige ZV}\\
X sei stetig verteilt mit Dichtefkt. $f_X$. Dann ist $E[X]=\int\limits^{+\infty}_{-\infty}x f_X(x)dx$, falls das Integral absolut konvergiert.

\begin{enumerate}
 \item Uniforme Verteilung ($X\sim U(0,1)$)\\
  $E[X]=\int\limits^1_0 xdx=[\frac{1}{2}x^2]^1_0=\frac{1}{2}$
 \item Normale Verteilung ($X \sim N(\mu,\sigma^2)$)\\
  $E[X]=\mu$
 \item Cauchy Verteilung\\
  $f(x)=\frac{1}{\Pi}\frac{1}{1+x^2},\: -\infty<x<+\infty$\\
  $E[X]=+\infty$
\end{enumerate}

\subsection{Funktionen von ZV}

\textbf{Satz}\\
Sei $Y=g(X)$.
Falls X diskret mit W'keitsfkt. p(x) ist, dann ist $E[g(X)]=\sum\limits_i g(x_i)p_x(x_i)$, falls die Summe absolut konvergiert.\\
Falls X stetig verteilt verteilt ist mit Dichtefkt. f, dann ist $E[g(X)]=\int\limits^{+\infty}_{-\infty}g(x)f(x)dx$ falls das Integral konvergiert.\\\\

\textbf{Satz}\\
Seien $X_1,\ldots,X_n,Y$ ZV mit $Y=g(X_1,\ldots,X_n)$. Falls $X_1,\ldots,X_n$ diskret sind mit $p(x_1,\ldots,x_n)$, dann $E[Y]=\sum\limits_{X_1}\ldots\sum\limits_{X_n}g(x_1,\ldots,x_n)p(x_1,\ldots,x_n)$\\
stetiger Fall:\\
$E[Y]=\int\limits^{+\infty}_{-\infty}g(x_1,\ldots,x_n)f(x_1,\ldots,x_n)dx_1\ldots dx_n$\\
falls das Integral absolut konvergiert.\\\\

\textbf{Korollar zum Satz}\\
Seien X,Y unabh. ZV\\
Dann $E[g(X)h(Y)]=E[g(X)]E[h(Y)]$

\subsection{Linearkombinationen von ZV}

\textbf{Satz}\\
Der E'wert ist ein linearer Operator, d.h. $X_1,\ldots,X_n$ seien ZV mit E'werten $E[X_1],\ldots,E[X_n]$. Sei $Y=a_i+\sum\limits^{n}_{i=1}b_i X_i$. Dann ist $E[Y]=a_i+\sum\limits^{n}_{i=1}b_i E[X_i]$.


\subsection{Varianz und Standardabweichung}

\textbf{Definition}\\
X sei eine ZV mit E'wert E[X]. Dann heisst $var(X)=E[(X-E[X])^2]$ die \textit{Varianz} von X (falls $var(X)<+\infty$). Es gilt aber auch $var(X)=E[X^2]-(E[X])^2$, was meist einfacher zu berechnen ist. Es gilt: $var(X)>0$!\\
$\sigma(X)=\sqrt{var(X)}$ heisst \textit{Standardabweichung}.\\

X diskret:\\
$var(X)=\sum\limits_{i}(x_i-\mu)^2p(x_i),\:\mu=E[X]$\\

X stetig:\\
$var(X)=\int\limits^{+\infty}_{-\infty}(x-\mu)^2f(x)dx,\:\mu=E[X]$\\

$E[X^2]=E[X]^2-var(X)$\\\\

\textbf{Satz}\\
Sei X eine ZV mit $var(X)<+\infty$, $a,b\in \mathbb{R}$. Dann ist $var(a+bX)=b^2 var(X)$\\

\begin{enumerate}
 \item Bernoulli-Verteilung $X\sim Be(p)$\\
  $E[X]=p$\\
  $var(X)=p(1-p)$
 \item Binomial Verteilung $X\sim B(n,p)$\\
  $E[X]=np$\\
  $var(X)=np(1-p)$
 \item Normalverteilung $X\sim N(\mu,\sigma)$\\
  $E[X]=\mu$\\
  $var(x)=\sigma^2$
 \item Uniforme Verteilung $U\sim U(0,1)$\\
  $E[X]=\frac{1}{2}$\\
  $var(U)=\frac{1}{12}$
 \item $X\sim Poisson(\mu)$\\
  $E[X]=\mu$\\
  $var(X)=\mu$
 \item Geometrische Verteilung\\
  $E[X]=\frac{1}{p}$\\
  $var(X)=\frac{1-p}{p^2}$
 \item Exponential Verteilung\\
  $E[X]=\frac{1}{\lambda}$\\
  $var(X)=\frac{1}{\lambda^2}$
\end{enumerate}

\textbf{Gamma Funktion}\\
$\alpha > 0\:\:\Gamma(\alpha)=\int\limits^{\infty}_{0}u^{\alpha-1}e^{-u}du$\\
$\alpha\in\mathbb{R}^{+}:\:\:\Gamma(\alpha)=(\alpha-1)!$\\
$\alpha > 0: \:\: \Gamma(\alpha)=(\alpha-1)\Gamma(\alpha-1)$\\
$\Gamma(\frac{1}{2})=\sqrt{\Pi}$, $\Gamma(\frac{3}{2})=\frac{1}{2}\sqrt{\Pi}$  


\subsection{Kovarianz und Korrelation}

$\mu_X = E[X],\:\mu_Y=E[Y]$\\
$cov(X,Y)=E[(x-\mu_X)(Y-\mu_Y)]$\\
$cov(X,Y)=E[XY]-\mu_X\mu_Y$\\

$E[XY]=\iint xy f(x,y)dxdy$ falls gem. Dichtefkt.\\
$E[XY]=E[X]\cdotp E[Y]$ falls unabh.\\

\textbf{Rechenregeln}
\begin{enumerate}
 \item X,Y unabh. $\Rightarrow\: cov(X,Y)=0$\\
  Die Umkehrung gilt aber nicht.
 \item $cov(X,X)=var(X)$
 \item $cov(aX,Y)=a\cdotp cov(X,Y)$
 \item $cov(X+Y,Z)=cov(X,Z)+cov(Y,Z)$
 \item $cov(X+Y,Z+W)=cov(X,Z)+cov(X,W)+cov(Y,Z)+cov(Y,W)$
 \item $cov(X,a)=0$ f"ur $a\in\mathbb{R}$
 \item $cov(a+\sum\limits^{n}_{i=1}b_i X_i,c+\sum\limits^{m}_{j=1}d_j Y_j)=\sum\limits^{n}_{i=1}\sum\limits^{m}_{j=1}b_i d_j cov(X_i,Y_j)$
 \item $var(a+\sum\limits^{n}_{i=1}b_iX_i)=\sum\limits^{n}_{i=1}\sum\limits^{n}_{j=1}b_i b_j cov(X_i,X_j)$
 \item $var(X+Y)=var(X)+var(Y)+2cov(X,Y)$
 \item Falls $X_1,\ldots,X_n$ unabh.:\\
  $var(\sum\limits^{n}_{i=1}X_i)=\sum\limits^{n}_{i=1}var(X_i)$\\
  im Gegensatz dazu gilt immer: $E[\sum\limits^{n}_{i=1}x_i]=\sum\limits^{n}_{i=1}E[X_i]$
\end{enumerate}

X, Y seien ZV mit endlichen Varianzen. Dann heisst\\
$P_{XY}=\rho(X,Y)=\frac{cov(X,Y)}{\sqrt{var(X) var(Y)}}$\\
die \textit{Korrelation}.\\\\

\textbf{Eigenschaften}
\begin{itemize}
 \item $\rho(X,Y) \in [-1,1]$
 \item falls $\rho(X,Y)=\pm 1$ dann $P[Y=aX+b]=1$ f"ur $a,b \in \mathbb{R},b\neq 0$
 \item $\rho(X,Y)\approx\pm 1\Rightarrow$ X und Y sind stark linear abh"angig
 \item $\rho(X,Y)\approx 0\Rightarrow$ X und Y sind schwach linear abh"angig
\end{itemize}


\section{Grenzwerts"atze}

Seien $X_1,\ldots,X_n$ unabh. ZV, mit $E[X_i]=\mu$ und $var(X_i)=\sigma^2<+\infty$.\\
Dann gilt:\\
$\lim\limits_{n\rightarrow\infty}P(\mid\bar{X}_n-\mu\mid\geq\epsilon)=0\:\:\forall\epsilon>0$


\subsection{Zentraler Grenzwertsatz}

$X_1,X_2,\ldots$ Folge identisch verteilter, unabh. ZV mit $E[X_i]=\mu,\: var(X_i)=\sigma^2<\infty$\\
$s_n=\sum\limits^{n}_{i=1}x_i$; Standardisierung: $\frac{s_n-\mu n}{\sqrt{n}\sigma}$\\

Standardisierung: $U=\frac{X-\mu}{\sigma}\sim N(0,1)$

\subsubsection{Zentraler Grenzwertsatz}
$\lim\limits_{n\rightarrow\infty}P(\frac{s_n-\mu n}{\sqrt{n}\sigma}\leq x)=\Phi(x)$, $\Phi$ ist die Vertfkt. der $N(0,1)$-Verteilung. Kann auch so geschrieben werden:\\
$\lim\limits_{n\rightarrow\infty}P(\frac{\bar{X}-\mu}{\sigma/\sqrt{n}}\leq x)\sim N(0,1)$\\\\

\textit{Beispiel} Gesucht ist x, so dass $P[\sum\limits^{120}_{i=1}I_i>x]=0.95$\\
Aus dem ZGS folgt:\\
$0.05=1-P[\sum\limits^{120}_{i=1}I_i>x]=P[\sum\limits^{120}_{i=1}I_i\leq x]=P[\frac{\sum\limits^{120}_{i=1}I_i-120E[I]}{\sqrt{120\:Var(I)}}\leq\frac{x-120E[I]}{\sqrt{120\:Var(I)}}]\approx\Phi(\frac{x-120E[I]}{\sqrt{120Var(I)}})$

\subsubsection{Monte-Carlo-Integration}
$j(f)=\int\limits^{1}_{0}f(x)dx$ sei numerisch zu berechnen\\
Generiere unabh. auf $[0,1]$ gleichverteilte ZV. $U_1,\ldots,U_n$ und berechne $\bar{f(U)}=\frac{1}{n}\sum\limits^{n}_{i=1}f(u_i)$.\\
Sei $var(f(U_i))$ endlich.\\
Nach dem Gesetz der grossen Zahlen gilt:\\
$\lim\limits_{n\rightarrow\infty}\frac{1}{n}\sum\limits^{n}_{i=1}f(U_i)=E[f(U_i)]=\int\limits^{1}_{0}dx$\\
$E[g(x)]=\int\limits^\infty_{-\infty}g(x)\cdotp f_x(x)dx$, wobei $f_x(x)$ die Dichte von x ist.

\subsection{Normalapproximation der Binomialverteilung}

Falls gilt $np(1-p)>9$ so kann $B(n;p)$ durch $N(\mu=np;\sigma=\sqrt{np(1-p)})$ approximiert werden, sonst falls $np\leq 10$ und $n\geq 1500p$ durch $Poiss(\lambda=np)$.\\

\textit{Beispiel}\\

VB auf $\alpha$ Niveau f"ur Binomialvrtlg.\\
$\frac{X-np}{\sqrt{np(1-p)}}\sim N(0,1)$\\
$\Rightarrow \{X>\Phi^{-1}(1-\alpha)\cdotp\sqrt{n\hat{p}(1-\hat{p})}+n\hat{p}\}$


\section{Statistik}

\textbf{Definition}\\
Eine \textit{Stichprobe} vom Umfang n ist eine Folge $X_1,\ldots,X_n$ von unabh., ident. verteilten ZV. Eine Statistik ist eine ZV $g(X_1,\ldots,X_n)$, wobei $g:\mathbb{R}^n\rightarrow \mathbb{R}$\\

\subsection{empirischer Mittelwert und empirische Varianz}

F"ur eine Stichprobe $X_1,\ldots,X_n,\:n\geq 2$

$$\bar{X}=\frac{1}{n}\sum\limits^{n}_{i=1}X_i$$
$$S^2=\frac{1}{n-1}\sum\limits^n_{i=1}(X_i-\bar{X})^2$$

\textbf{Satz}\\
Falls die $X_i$ EW $\mu$ und $var\: \sigma^2$ haben, so ist $E[\bar{X}]=\mu,\:var(\bar{X})=\frac{\sigma^2}{n}$. $\bar{X}$ ist eine ZV, $\mu$ eine Zahl. Man sagt, $\bar{X}$ ist ein Sch"atzer von $\mu$.\\

\textbf{Satz}\\
Falls die $X_i$ $N(\mu,\sigma^2)$-verteilt sind, so ist $\bar{X}$ $N(\mu,\frac{\sigma^2}{n})$-verteilt.\\

\textbf{Satz}\\
Sei $var(X_i)=\sigma^2$. Dann $E[S^2]=\sigma^2$. Man sagt, der Sch"atzer $S^2$ habe keinen Bias.

\subsection{$\chi^2$ Verteilung}

\textbf{Satz}\\
Falls $X\sim N(0,1)$, dann ist $X^2\sim\Gamma(\frac{1}{2},\frac{1}{2})$. Diese Verteilung nennt man $\chi^2_1$-Verteilung. Lies: chi-Quadrat mit 1 Freiheitsgrad.\\

\textbf{Satz}\\
Seien $X_1,\ldots,X_n$ unabh. ident. verteilt, $X_i\sim\chi_1^2$. Dann ist $V=X_1+\ldots+X_n$ $\chi^2_n$-verteilt. $\chi^2_n$ ist eine $\Gamma(\frac{1}{2},\frac{1}{2})$-Verteilung. D.h., die Dichte ist\\
$f(t)=\frac{1}{\Gamma(\frac{n}{2}2^{n/2}}t^{\frac{n}{2}-1}e^{-\frac{t}{2}}\:\:t\geq 0$\\
Falls $V\sim \chi^2_n$, dann $E[V]=n,\:var(V)=2n$\\

\textbf{Satz}\\
Seien $X_1,\ldots,X_n$ eine Stichprobe von $N(\mu,\sigma^2)$-verteilten ZV. Dann ist $\frac{(n-1)S^2}{\sigma^2}\sim\chi^2_{n-1}$

\subsection{t Verteilung}

\textbf{Satz}\\
Sei $X_1,\ldots,X_n$ eine Stichprobe aus einer $N(\mu,\sigma^2)$-Population. Dann ist $t=\frac{\bar{X}-\mu}{S/\sqrt{n}}\:t_{n-1}$ verteilt. Dichte der t-Verteilung: $f(x)=\frac{\Gamma(\frac{n+1}{2})}{\sqrt{\Pi n}\Gamma(\frac{1}{2})}\cdotp(1+\frac{x^2}{n})^{\frac{-n+1}{2}}$.

\subsection{F Verteilung}

U und V unabh. $\chi^2$ ZV mit m bzw. n Freiheitsgraden, so wird die Verteilung:

$$W=\frac{U/m}{V/n}$$

als F Verteilung mit m und n Freiheitsgraden bezeichnet, geschrieben $F_{m,n}$

\section{Konfidenzintervalle ...}

\subsection{... f"ur unbekannte Mittelwert $\mu$ einer Normalverteilung bei bekannter Varianz $\sigma^2$}

Vertrauensniveau $(1-\alpha)$ w"ahlen (z.B. 0.95)

\subsubsection{Eine Stichprobe}

\begin{enumerate}
 \item Verteilung ist gleich $Z=\frac{\bar{X}-\mu}{\sigma/\sqrt{n}}\sim N(0,1)$
 \item Konfidenzintervall
  \begin{enumerate}
   \item \textit{zweiseitig}\\
    $P[q_{\frac{\alpha}{2}}\leq\frac{\bar{X}-\mu}{\sigma/\sqrt{n}}\leq q_{1-\frac{\alpha}{2}}]=1-\alpha$\\
    $\Rightarrow [\bar{X}- q_{1-\frac{\alpha}{2}}\cdotp\frac{\sigma}{\sqrt{n}}\leq\mu\leq\bar{X}+q_{\frac{\alpha}{2}}]$
   \item \textit{nach oben}\\
    $P[\frac{\bar{X}-\mu}{\sigma/\sqrt{n}}\leq q_{1-\alpha}]=1-\alpha$
   \item \textit{nach unten}\\
    $P[q_{\alpha}\leq\frac{\bar{X}-\mu}{\sigma/\sqrt{n}}]=1-\alpha$
  \end{enumerate}
\end{enumerate}


\subsubsection{Zwei Stichproben (Differenz der Mittelwerte)}

gleich wie eine Stichprobe, aber mit folgender Verteilung:

$$\frac{(\bar{X}-\bar{Y})-(\mu_X-\mu_Y)}{\sqrt{\frac{\sigma_X^2}{m}+\frac{\sigma_Y^2}{n}}}\sim N(0,1)$$

\subsection{... f"ur unbekannten Mittelwert $\mu$ einer Normalverteilung bei unbekannter Varianz $\sigma^2$}

Vertrauensniveau $(1-\alpha)$ w"ahlen (z.B. 0.95)

\subsubsection{Eine Stichprobe}

\begin{enumerate}
 \item Verteilung ist gleich $T=\frac{\bar{X}-\mu}{S/\sqrt{n}}\sim t_{n-1}$
 \item Konfidenzintervall
  \begin{enumerate}
   \item \textit{zweiseitig}\\
    $P[t_{n-1,\frac{\alpha}{2}}\leq T \leq t_{n-1,1-\frac{\alpha}{2}}]=1-\alpha$\\
    $\Rightarrow [\bar{X_n}-t_{n-1,1-\frac{\alpha}{2}}\cdotp\frac{S}{\sqrt{n}}\leq \mu \leq \bar{X_n}+t_{n-1,1-\frac{\alpha}{2}}\cdotp\frac{S}{\sqrt{n}}]$
   \item \textit{nach oben}\\
    $P[T \leq t_{n-1,1-\alpha}]=1-\alpha$
   \item \textit{nach unten}\\
    $P[t_{n-1,\alpha}\leq T]=1-\alpha$
  \end{enumerate}
\end{enumerate}

\subsubsection{Zwei Stichproben (Differenz der Mittelwerte)}

gleich wie eine Stichprobe, aber mit folgender Verteilung:
$$\frac{(\bar{X}-\bar{Y})-(\mu_X-\mu_Y)}{S_P\sqrt{\frac{1}{m}+\frac{1}{n}}}\sim t_{m+n-2}$$

wobei $S_p$:
$$S_p^2=\frac{(m-1)S_X^2+(n-1)S_Y^2}{m+n-2}$$
die gepoolte Varianz ist\\

falls m=n und Stichproben nicht unbedingt unabh.: $D_i=X_i-Y_i$\\
$\frac{\bar{D}-(\mu_X-\mu_Y)}{S_D/\sqrt{n}}\sim t_{n-1}$

\subsection{... f"ur unbekannte Varianz $\sigma^2$ einer Normalverteilung}

Vertrauensniveau $(1-\alpha)$ w"ahlen (z.B. 0.95)

\subsubsection{Eine Stichprobe}

$$\frac{S^2}{\sigma^2}(n-1)\sim \chi^2_{n-1}$$

\subsubsection{Zwei Stichproben}

$$\frac{S_X^2/\sigma_X^2}{S_Y^2/\sigma_Y^2}\sim F_{(m-1,n-1)}$$

Wobei m (n) Anz. Experimente f"ur ZV X (Y).

\subsection{... f"ur unbekannten Parameter p einer Binomialverteilung}

\begin{enumerate}
 \item Verteilung ist gleich $Z=\frac{n\hat{p}-np}{\sqrt{n\hat{p}(1-\hat{p})}}\sim N(0,1)$, wobei $\hat{p}=\frac{k}{n}$ k= Anz. Erfolge bei n Versuchen.
 \item Konfidenzintervall
  $[\hat{p}-\frac{q_{1-\frac{\alpha}{2}}}{n}\sqrt{n\hat{p}(1-\hat{p})} \leq p \leq \hat{p}-\frac{q_{\frac{\alpha}{2}}}{n}\sqrt{n\hat{p}(1-\hat{p})}]$
\end{enumerate}


\section{Sch"atztheorie}

Ein Sch"atzer ist \textit{erwartungstreu}, falls $E[\hat{\theta}]=\theta$.

\subsection{Maximum Likelihood}

Sei $X_1,\ldots,X_n$ eine Stichprobe des Umfangs n einer Dichte $f(x,\theta)$, dann ist die gemeinsame Dichte von $(X_1,\ldots,X_n)$ die Likelihood-Funktion.
\begin{itemize}
 \item X diskret:\\
  $L(\Theta)=\prod\limits^n_{i=1}P(X_i=x_i)$
 \item X stetig:\\
  $L(\Theta)=\prod\limits^n_{i=1}f(x_i)$
\end{itemize}

W"ahle $\Theta$ so dass die Realisierungen $X_1,\ldots,X_n$ am wahrscheinlichsten sind. Um die Berechnung zu vereinfachen: logarithmiere, so ergibt sich aus dem Produkt eine Summe. $l(\Theta)=\log L(\Theta)$. Differenziere danach und setze gleich 0. $\frac{\partial}{\partial\Theta}l(\Theta)\stackrel{!}{=}0$. L"ose danach nach $\Theta \Rightarrow\:\hat{\Theta}_{ML}$


\subsection{Momentenmethode}

Berechne f"ur $X\sim F(\Theta)$:
\begin{itemize}
 \item $E[X]$ h"angt von $\Theta$ ab (da $f(x)$ oder $P(X=x_i)$ von $\Theta$ abh"angen).\\
  $E[X]=\int^\infty_{-\infty}xf(x)dx$ oder $E[X]=\sum\limits_{i=1}^n x_iP(X=x_i)$
 \item $\bar{X}=\frac{1}{n}\sum\limits^n_{i=1}X_i$
\end{itemize}

Danach setze $E[X]=\bar{X}$ und l"ose nach $\Theta\Rightarrow\:\hat{\Theta}_{MM}$\\

p-tes Moment: Setze auch noch $E[X^2]=\bar{X^2}\ldots E[X^p]=\bar{X^p}\:\Rightarrow$ p-Gleichungen, l"ose nach $\Theta$

\section{Testen von Hypothesen}

\subsection{Neyman-Pearson Paradigma}

\textbf{Nullhypothese $H_0$}:\\
Die zu zeigende Aussage, meist also $\mu=\mu_0$\\

\textbf{Alternative $H_A$}:\\
Was gilt, falls $H_A$ nicht gilt:\\
$\mu\neq\mu_0,\:\mu>\mu_0\:,\mu<\mu_0$\\

\textbf{Fehler 1. Art ($\alpha$)}:\\
Verwerfung von $H_0$, obwohl $H_0$ richtig. $\alpha=P(Fehler\:1.Art)$\\

\textbf{Fehler 2. Art ($\beta$)}:\\
Keine Ablehnung von $H_0$, obwohl $H_0$ falsch. $\beta=P(Fehler\:2.Art)$.\\

\textbf{Macht}
Macht = $1-\beta$. W'keit, dass $H_0$ verworfen wird, wenn es tats"achlich falsch ist.\\

Man versucht $\alpha$ m"oglichst klein, und $1-\beta$ m"oglichst gross zu w"ahlen. Dazu wird $\alpha$ fixiert. Danach konstruiert man dazu einen Test mit m"oglichst grosser Macht.\\

\textit{Beispiel} Macht, wenn $\mu_X,\sigma_X$ gleich wirklicher Mittelwert bzw. Varianz und $G$ die zuvor ausgerechnete Grenze des VB's ist:\\
$Macht=1-P(Fehler\:2.\:Art)=1-P(X\leq G)=1-P(\frac{X-\mu_x}{\sigma_X}\leq\frac{G-\mu_x}{\sigma_x})=1-\Phi(\frac{G-\mu_x}{\sigma_x})$

\subsection{Neyman-Pearson Lemma}

Likelihood ratio = $\frac{f_0(x)}{f_A(x)}$\\
$\rho_L=\left\{
 \begin{array}{l}
  $1 wenn $\frac{f_0(x)}{f_A(x)}<K_{\alpha}\\
  $0 wenn $\frac{f_0(x)}{f_A(x)}>K_{\alpha}
 \end{array}
\right.$\\
$K_{\alpha}$ muss so gew"ahlt sein, dass $E_0[\rho_L]=\alpha$\\

Der Likelihood-Test $\rho_L$ ist der m"achtigste Test unter den Tests $\rho^*$ mit Signifikanzlevel $\alpha^*\leq \alpha$

\subsection{Der t-Test}

Mittelwert bei unbekanntem $\sigma$\\

\textbf{Modellannahmen}:
\begin{itemize}
 \item $X_i \sim N(\mu,\sigma)$
 \item $X_i$ unabh"angig
 \item $\mu,\sigma$ unbekannt
\end{itemize}

Wir wollen testen, ob $\mu=\mu_0$ oder Alternative.\\\\

\textbf{F"ur den t-Test gilt}:\\
$T=\sqrt{n}\frac{\bar{X}-\mu_0}{S}$\\

Unter $H_0$: $T\sim t_{n-1}$\\

\textbf{Verwerfungsbereich}:
\begin{itemize}
 \item $H_A:\:\mu\neq\mu_0$\\
  $VB=\{\mid T \mid \geq t_{n-1,1-\frac{\alpha}{2}}\}$
 \item $H_A:\:\mu>\mu_0$\\
  $VB=\{T > t_{n-1,1-\alpha}\}$
 \item $H_A:\:\mu<\mu_0$\\
  $VB=\{T < t_{n-1,\alpha}\}$
\end{itemize}

Falls $T$ bzw. $|T|$ in VB $\Rightarrow$ $H_0$ verwerfen, sonst $H_0$ beibehalten.


\subsubsection{2-Stichproben t-Test}

$T=\frac{(\bar{X}-\bar{Y})-(\mu_X-\mu_Y)}{S_P\sqrt{\frac{1}{m}+\frac{1}{n}}}\sim t_{m+n-2}$. Wobei $S_p^2=\frac{(m-1)S_X^2+(n-1)S_Y^2}{m+n-2}$ die gepoolte Varianz ist.\\

\subsection{Der z-Test}

Mittelwert bei bekanntem $\sigma$\\

\textbf{Modellannahmen}:
\begin{itemize}
 \item $X_i \sim N(\mu,\sigma)$
 \item $X_i$ unabh"angig
 \item $\mu$ unbekannt, $\sigma$ bekannt
\end{itemize}

Wir wollen testen, ob $\mu=\mu_0$ oder Alternative.\\\\

\textbf{Verteilung}:\\
$\frac{\bar{X}-\mu}{\sigma/\sqrt{n}}\sim N(0,1)$\\

\textbf{Verwerfungsbereich}:
\begin{itemize}
 \item $H_A:\:\mu\neq\mu_0$\\
  $VB=\{|\frac{\bar{X}-\mu_0}{\sigma /\sqrt{n}}|>q_{1-\frac{\alpha}{2}}\}$
 \item $H_A:\:\mu>\mu_0$\\
  $VB=\{\frac{\bar{X}-\mu_0}{\sigma /\sqrt{n}}>q_{1-\alpha}\}$
 \item $H_A:\:\mu<\mu_0$\\
  $VB=\{\frac{\bar{X}-\mu_0}{\sigma /\sqrt{n}}<q_{\alpha}\}$
\end{itemize}


\subsection{Likelihood-Ratio Tests f"ur Multinomialverteilungen}

m Zellen, n Beobachtungen $\Rightarrow$ Histogramm\\

$-2\ln\Lambda=2\sum\limits^m_{i=1}x_i\ln(\frac{x_i}{E_i})$ ist $\chi^2$-verteilt mit $m-1-k$ Freiheitsgraden, wobei k die Anz. freier Parameter in $H_0$ ist.

\subsubsection{$\chi^2$ Anpassungstest}

\begin{itemize}
 \item Modellannahme: $X \sim F$, F irgend eine Verteilungsfkt.
 \item Nullhypothese $H_0:\:F=Pois(\lambda)$
 \item Alternative $H_A:\:F\neq Pois(\lambda)$
 \item Die Teststatistik ist gegeben durch die $\chi^2$ Teststatistik: man bildet die qudrierten Differenzen zwischen den beobachteten H"aufigkeiten ($Beob_i$) und den erwarteten H"aufigkeiten ($Erw_i$), man teilt durch die erwarteten H"aufigkeiten ($Erw_i$) und summiert "uber alle m"oglichen Klassen.\\
  n= Anz. Klassen\\
  $\chi^2=\sum\limits^n_{i=0}\frac{(Beob_i-Erw_i)^2}{Erw_i}$\\
  Unter $H_0$ ist die Teststatistik $\chi^2$-verteilt mit $f$ Freiheitsgraden, wobei\\
  f= Anz. Klassen - 1 - Anzahl gesch"atzter Parameter
 \item Entscheidung:\\
  VB=$\{\chi^2>\chi^2_{f;p}\}$
\end{itemize}

\subsubsection{Vorzeichen Test}


\begin{tabular}{ll}
 Modellannahme:&	$D_i$ unabh"angig mit Median $\tilde{m}$\\
 Nullhypothese:&	$\tilde{m}=0$\\
 Alternative:&		$\tilde{m}\neq 0$\\
 Teststatistik:&	$T=\sum\limits^{n}_{i=1}1|_{\{D_i>0\}}$\\
 Unter $H_0$:&		$T\sim Bin(n,p)$
\end{tabular}\\

\textit{Beispiel} f"ur $T=7$ und $n=10$\\
$2 P[T\geq 7]=2\sum\limits^{10}_{i=1}P[T=i]=0.34$\\
Da $0.34>0.05$ wird $H_0$ beibehalten auf 5\% Test

\section{Methode der kleinsten Quadrate und lineare Regression}

\textbf{Beobachtungen}:\\
$\{(x_i,y_i)\mid i=1\ldots n\}$\\
y = die abh"angige Variabel, ''zu erkl"arende Variabel''\\
x = die unabh"angige Variabel, ''erkl"arende Variabel''\\

\textbf{Ansatz}:\\
Der Zusammenhang zw. x und y ist linear, d.h $y=\beta_0+\beta_1 x$\\

Abweichung (Residuum) des i-ten Punktes:\\
$e_i=y_i-(\beta_0+\beta_1 x_i)$\\

\textbf{Ergebnis}:\\
$\hat{\beta_0}=\frac{(\sum\limits^n_{i=1}x_i^2)(\sum\limits^n_{i=1}y_i)-(\sum\limits^n_{i=1}x_i)(\sum\limits^n_{i=1}x_iy_i)}{n\sum\limits^n_{i=1}x_i^2-(\sum\limits^n_{i=1}x_i)^2}$\\
$\hat{\beta_1}=\frac{\sum\limits^n_{i=1}(x_i-\bar{X})(y_i-\bar{Y})}{\sum\limits^n_{i=1}(x_i-\bar{X})^2}$

\subsection{Statistisches Modell}

$y_i=\beta_0+\beta_1 x_i+e_i$\\
$e_i$: Beobachtungsfehler\\

\textbf{Voraussetzungen}:
\begin{enumerate}
 \item $e_i$ sind unabh"angig
 \item $E[e_i]=0$
 \item $var(e_i)=\sigma^2$
 \item $x_i$ sind fest
\end{enumerate}

\textbf{Varianz}\\
$var(\hat{\beta_0})=\frac{\sigma^2\sum\limits^n_{i=1}x_i^2}{n\sum\limits^n_{i=1}x_i^2-(\sum\limits^n_{i=1}x_i)^2}$\\
$var(\hat{\beta_1})=\frac{n\sigma^2}{n\sum\limits^n_{i=1}x_i^2-(\sum\limits^n_{i=1}x_i)^2}=\frac{\sigma^2}{\sum\limits^n_{i=1}(x_i-\bar{X})^2}$\\


\textbf{Korrelation}\\
$cov(\hat{\beta_0},\hat{\beta_1})=\frac{-\sigma^2\sum\limits^n_{i=1}x_i}{n\sum\limits^n_{i=1}x_i^2-(\sum\limits^n_{i=1}x_i)^2}$\\

\textbf{Bemerkungen}
\begin{itemize}
 \item $\hat{e_i}:=y_i-\hat{\beta_0}-\hat{\beta_1}x_i$\\
 \item $S^2=\frac{\sum\limits^n_{i=1}(\hat{e_i})^2}{n-2}$ ist bias-freier Sch"atzer f"ur $\sigma^2$
 \item Setze $S^2$ ein in Formel f"ur $var(\hat{\beta_0})$ bzw. $var(\hat{\beta_1})$ und erhalte: $s_{\hat{\beta_0}}^2$ bzw. $s_{\hat{\beta_1}}^2$.
 \item Wenn $e_i$ normalverteilt sind, dann sind $\hat{\beta_0}$, $\hat{\beta_1}$ auch normalverteilt, und $\frac{\hat{\beta_0}-\beta_0}{s_{\hat{\beta_0}}}$ und $\frac{\hat{\beta_1}-\beta_1}{s_{\hat{\beta_1}}}$ sind t-verteilt mit n-2 Freiheitsgraden.
\end{itemize}

\subsubsection{Korrelation und Regression}

$S_{XX}=\frac{1}{n}\sum\limits^n_{i=1}(x_i-\bar{X})^2$ ''Varianz von X''\\
$S_{YY}=\frac{1}{n}\sum\limits^n_{i=1}(y_i-\bar{Y})^2$ ''Varianz von Y''\\
$S_{XY}=\frac{1}{n}\sum\limits^n_{i=1}(x_i-\bar{Y})(y_i-\bar{Y})$ ''Kovarianz von X und Y''\\\\

Der Korrelationskoeffizient von X und Y ist:\\
$r=\frac{S_{XY}}{\sqrt{S_{XX}S_{YY}}}$


\end{document}
